\documentclass[
11pt, % The default document font size, options: 10pt, 11pt, 12pt
%codirector, % Uncomment to add a codirector to the title page
]{charter} 




% El títulos de la memoria, se usa en la carátula y se puede usar el cualquier lugar del documento con el comando \ttitle
\titulo{Controlador CAN de Servomotores} 

% Nombre del posgrado, se usa en la carátula y se puede usar el cualquier lugar del documento con el comando \degreename
\posgrado{Carrera de Especialización en Sistemas Embebidos} 
%\posgrado{Carrera de Especialización en Internet de las Cosas} 
%\posgrado{Carrera de Especialización en Intelegencia Artificial}
%\posgrado{Maestría en Sistemas Embebidos} 
%\posgrado{Maestría en Internet de las cosas}

% Tu nombre, se puede usar el cualquier lugar del documento con el comando \authorname
\autor{Alejandro Virgillo} 

% El nombre del director y co-director, se puede usar el cualquier lugar del documento con el comando \supname y \cosupname y \pertesupname y \pertecosupname
\director{Gabriel Gavinowich}
\pertenenciaDirector{FIUBA} 

\codirector{} % para que aparezca en la portada se debe descomentar la opción codirector en el documentclass
\pertenenciaCoDirector{}

% Nombre del cliente, quien va a aprobar los resultados del proyecto, se puede usar con el comando \clientename y \empclientename
\cliente{Alejandro Virgillo}
\empresaCliente{A3 Engineering}

% Nombre y pertenencia de los jurados, se pueden usar el cualquier lugar del documento con el comando \jurunoname, \jurdosname y \jurtresname y \perteunoname, \pertedosname y \pertetresname.
\juradoUno{Nombre y Apellido (1)}
\pertenenciaJurUno{pertenencia (1)} 
\juradoDos{Nombre y Apellido (2)}
\pertenenciaJurDos{pertenencia (2)}
\juradoTres{Nombre y Apellido (3)}
\pertenenciaJurTres{pertenencia (3)}
 
\fechaINICIO{21 de octubre de 2021}		%Fecha de inicio de la cursada de GdP \fechaInicioName
\fechaFINALPlan{9 de diciembre de 2021} 	%Fecha de final de cursada de GdP
\fechaFINALTrabajo{9 de octubre de 2022}	%Fecha de defensa pública del trabajo final


\begin{document}

\maketitle
\thispagestyle{empty}
\pagebreak


\thispagestyle{empty}
{\setlength{\parskip}{0pt}
\tableofcontents{}
}
\pagebreak


\section*{Registros de cambios}
\label{sec:registro}


\begin{table}[ht]
\label{tab:registro}
\centering
\begin{tabularx}{\linewidth}{@{}|c|X|c|@{}}
\hline
\rowcolor[HTML]{C0C0C0} 
Revisión & \multicolumn{1}{c|}{\cellcolor[HTML]{C0C0C0}Detalles de los cambios realizados} & Fecha      \\ \hline
0      & Creación del documento                                 &\fechaInicioName \\ \hline
1      & Se completa hasta el punto 5 inclusive                 & 30 de octubre de 2021 \\ \hline
2      & Se completa hasta el punto 9 inclusive
& 7 de noviembre de 2021 \\ \hline
%3      & Se completa hasta el punto 11 inclusive                & dd/mm/aaaa \\ \hline
%4      & Se completa el plan	                                 & dd/mm/aaaa \\ \hline
\end{tabularx}
\end{table}

\pagebreak



\section*{Acta de constitución del proyecto}
\label{sec:acta}

\begin{flushright}
Buenos Aires, \fechaInicioName
\end{flushright}

\vspace{2cm}

Por medio de la presente se acuerda con el Ing. \authorname\hspace{1px} que su Trabajo Final de la \degreename\hspace{1px} se titulará ``\ttitle'', consistirá esencialmente en la implementación de una interfaz que permita configurar y relevar información de servomotores conectados a través de una red CAN, y tendrá un presupuesto preliminar estimado de 600 hs de trabajo y U\$1000, con fecha de inicio \fechaInicioName\hspace{1px} y fecha de presentación pública \fechaFinalName.

Se adjunta a esta acta la planificación inicial.

\vfill

% Esta parte se construye sola con la información que hayan cargado en el preámbulo del documento y no debe modificarla
\begin{table}[ht]
\centering
\begin{tabular}{ccc}
\begin{tabular}[c]{@{}c@{}}Ariel Lutenberg \\ Director posgrado FIUBA\end{tabular} & \hspace{2cm} & \begin{tabular}[c]{@{}c@{}}\clientename \\ \empclientename \end{tabular} \vspace{2.5cm} \\ 
\multicolumn{3}{c}{\begin{tabular}[c]{@{}c@{}} \supname \\ Director del Trabajo Final\end{tabular}} \vspace{2.5cm} \\
%\begin{tabular}[c]{@{}c@{}}\jurunoname \\ Jurado del Trabajo Final\end{tabular}     &  & \begin{tabular}[c]{@{}c@{}}\jurdosname\\ Jurado del Trabajo Final\end{tabular}  \vspace{2.5cm}  \\
%\multicolumn{3}{c}{\begin{tabular}[c]{@{}c@{}} \jurtresname\\ Jurado del Trabajo Final\end{tabular}} \vspace{.5cm}                                                                     
\end{tabular}
\end{table}




\section{1. Descripción técnica-conceptual del proyecto a realizar}
\label{sec:descripcion}

El proyecto busca obtener un dispositivo que permita programar y supervisar servomotores conectados dentro de una red CAN (\textit{Controller Area Network}), así como relevar información de estos. El resultado debe ser de carácter industrial, por lo que se prioriza su robustez para poder operar en planta y debe contar con una interfaz de usuario.

La organización \empclientename{} desarrolló un sistema embebido, llamado SN-17, capaz de controlar la posición, velocidad, aceleración y torque a motores del tipo paso a paso, también conocidos como \textit{steppers}. Actualmente, varios de estos motores, junto con las placas controladoras mencionadas, se encuentran en funcionamiento en la planta de la empresa Cambre, realizando diversos tipos de actuaciones mecánicas en líneas de manufactura.

El sistema SN-17 cuenta con un problema, que es la dificultad para alterar, de forma cómoda, el funcionamiento de los motores. El firmware debe ser modificado cada vez que se quiera realizar un cambio de este estilo, lo cual limita la operabilidad. 

Dentro de este contexto es que se propone el actual proyecto. Se buscará lograr una interfaz que permita a un usuario con poco entrenamiento conectarse con los motores y modificar los parámetros de funcionamiento. Para ello, se establecerá comunicación a través de una red CAN, aprovechando que el sistema SN-17 cuenta con un puerto para este protocolo. Es necesario, entonces, diseñar un sistema de mensajeria que haga que los dispositivos puedan enviar información dentro de la red, y una interfaz gráfica que permita a usuarios realizar cambios.

También se propone que el dispositivo actue como supervisor, una vez que los motores estén en funcionamiento. Como en la mayoría de procesos industriales suelen emplearse controladores PLC (\textit{Programmable Logic Controller}), es necesario que la solución buscada pueda interactuar con estos.

En la \textbf{Figura 1} se muestra, a modo de ejemplo, un diagrama en bloques del sistema. El proyecto abarca el diseño y fabricación de la parte que aparece denominada como controlador y su interacción con los demás bloques. También notar que en el bus CAN pueden haber conectados más de 1 servomotor. Cada uno de estos tendrá una de las placas controladoras descriptas previamente.

\begin{figure}[htpb]
\centering 
\includegraphics[width=.8\textwidth]{./Figuras/CAN_Servo_controller.png}
\caption{Diagrama en bloques del sistema}
\label{fig:diagBloques}
\end{figure}


\section{2. Identificación y análisis de los interesados}
\label{sec:interesados}

\begin{table}[ht]
%\caption{Identificación de los interesados}
%\label{tab:interesados}
\begin{tabularx}{\linewidth}{@{}|l|X|X|l|@{}}
\hline
\rowcolor[HTML]{C0C0C0} 
Rol           & Nombre y Apellido & Organización 	& Puesto 	\\ \hline
Auspiciante   & \clientename      & \empclientename	&       Líder de proyecto 	\\ \hline
Cliente       & \clientename      &\empclientename	&       Líder de proyecto 	\\ \hline
Impulsor      &Sector de automatización                   &  Cambre            	&    -    	\\ \hline
Responsable   & \authorname       & FIUBA        	& Alumno 	\\ \hline
Colaboradores &  Andres Battisti                 &             Cambre 	&   Jefe de Automatización     	\\ \hline
Orientador    & \supname	      & \pertesupname 	& Director Trabajo final \\ \hline
Usuario final &     Sector de automatización          &Cambre              	& Técnicos        	\\ \hline
Usuario final &     Sector de armado          &Cambre              	& Operarios        	\\ \hline
\end{tabularx}
\end{table}


Por ejemplo:
\begin{itemize}
	\item Orientador: Gabriel Gavinowich, es especialista en sistemas embebidos y trabaja en protocolo CAN, será de gran ayuda en materias de este aspecto.
	\item Colaborador: Andrés Battisti, es hábil en la coordinación de proyectos. Puede colaborar con la implementación en planta.
	\item Usuario final: Sería de ayuda consultar a los operarios del sector de automatización para determinar temas de uso y de calidad que puedan considerar deseable. Pueden dar pautas para requerimientos que pueden ser de utilidad.
\end{itemize}


\section{3. Propósito del proyecto}
\label{sec:proposito}

El propósito de este proyecto es desarrollar un sistema embebido que actúe de interfaz para configurar y supervisar servomotores conectados a una red CAN.

\section{4. Alcance del proyecto}
\label{sec:alcance}

El proyecto incluye:

\begin{itemize}
	\item Una interfaz de usuario que permite configurar y supervisar los servomotores conectados.
	\item La estructura de mensajes que ha de transmitirse a través del bus CAN.
	\item La inclusión de entradas y salidas eléctricamente aisladas para comunicación con un PLC.
	\item La configuración de la red CAN.
	\item El desarrollo y fabricación de una plaqueta que abarque al sistema.
\end{itemize}

El proyecto no incluye:

\begin{itemize}
	\item El acceso a los datos de funcionamiento de los servomotores de forma remota o el almacenamiento de estos en una memoria.
	\item El desarrollo de las placas controladoras de los servomotores.
	\item La implementación final en planta.
\end{itemize}

\section{5. Supuestos del proyecto}
\label{sec:supuestos}

Para el desarrollo del presente proyecto se supone que:

\begin{itemize}
	\item El dinero disponible será suficiente para la adquisición de los materiales requeridos.
	\item Habrá stock de los componentes del sistema y no habrá problemas de importación. 
	\item Las demoras para obtener los componentes no serán excesivas.
	\item La relación con la empresa Cambre se mantendrá durante el transcurso del proyecto.
\end{itemize}


\section{6. Requerimientos}
\label{sec:requerimientos}
\newcommand{\leadingZeroes}[1]{%
\ifnum #1<100{0}\fi%
\ifnum #1<10{0}\fi%
#1}
%
\newcounter{REQ}
%
\newcommand{\REQ}{%
\stepcounter{REQ}%
\textbf{[SCI-CAN-REQ{\leadingZeroes{\theREQ}]}}}
%

\begin{enumerate}
	\item \textbf{Requerimientos de la red CAN:}
	\begin{enumerate}
			\item El sistema debe comunicarse empleando el protocolo CAN. \REQ
			\item El sistema debe poder comunicarse con todos los motores conectados a la red CAN que posean la placa controladora SN-17. \REQ
			\item El sistema debe poder comunicarse con hasta 5 motores conectados a la red CAN. \REQ
			\item El sistema debe envíar y recibir información a través del puerto CAN a una tasa de al
menos 300 kbps. \REQ
	\end{enumerate}
	\item \textbf{Requerimientos de comunicación:}
	\begin{enumerate}
		\item El sistema debe poder enviar programas a los servomotores de hasta 15 instrucciones
de largo. \REQ
		\item El sistema debe tener la capacidad de enviar cada una de las instrucciones de
programa disponibles en los servomotores. Estas son: Seteo de tipo de control,
verificación de señal de control, pausa de la señal de control, apagado de control,
comunicación, reseteo de programa. \REQ
		\item El sistema debe enviar, para cada instrucción, todos los atributos de configuración
necesarios. Estos son: Selección de variable de control, seteo de variable de control,
tiempo de cumplimiento de instrucción, tiempo de timeout, error permisible de la
variable de control, torque máximo de motor, tipo de comunicación y mensaje. \REQ
		\item El sistema debe poder modificar los parámetros de los lazos de control PID de los
servomotores para los distintos tipos de control. \REQ
		\item El sistema debe poder configurar el seteo de funcionalidades especiales de los
motores, estas son: selección de tipo de cerado, proceso de cerado, calibración del encoder, calibración de posiciones, forma de operatividad de las salidas discretas de los motores, verificación de detención de eje. \REQ
		\item El sistema debe permitir operar los motores en forma manual, simulando las entradas
y forzando las salidas discretas que cada uno de los motores conectados. \REQ
		\item El sistema deberá poder activar o desactivar a cada uno de los motores conectados. \REQ
		\item El sistema deberá recibir de cada motor conectado el número de
programa e instruccion en que se encuentra. \REQ
		\item El sistema deberá recibir de cada motor conectado un reporte de
error en caso de falla. \REQ
	\end{enumerate}
	\item \textbf{Requerimientos de comunicación con PLC:}
	\begin{enumerate}
		\item El sistema debe poder comunicarse con un controlador externo (PLC) a través de señales discretas. \REQ
		\item Las señales al controlador externo deben estar eléctricamente aisladas, por lo menos a 1 kV, usando optoacopladores. \REQ
		\item Las señales al controlador externo deben ser del tipo NPN. \REQ
		\item Las señales al controlador externo deben poder ser a distinta tensión que la empleada por el microcontrolador, hasta 24V. \REQ
		\item El sistema deberá enviar señales discretas en caso de que alguno de los motores
conectados reporte un error. \REQ
	\end{enumerate}
	\item \textbf{Requerimientos de interfaz de usuario:}
	\begin{enumerate}
		\item El sistema debe tener una pantalla LCD y una botonera. \REQ
		\item La pantalla LCD debe permitir acceder a las variables de los motores a través de un menú y debe recorrerse usando la botonera. \REQ
		\item El sistema debe tener un switch que permita cambiar su forma de funcionamiento, entre configurador de motores y relevador de información. \REQ
	\end{enumerate}
	\item \textbf{Requerimientos de diseño:}
	\begin{enumerate}
		\item El sistema debe emplear el microcontrolador ATSAMC21. \REQ
		\item El sistema debe alimentarse con 24V de tensión. \REQ
		\item El sistema debe estar encapsulado dentro de una caja plástica. \REQ
	\end{enumerate}
\end{enumerate}

\section{7. Historias de usuarios (\textit{Product backlog})}
\label{sec:backlog}

En esta sección se enuncian las historias de usuario, cada una de ellas llevará un puntaje según 3 aspectos:
\begin{itemize}
	\item Dificultad: Cantidad de trabajo a realizar.
	\item Complejidad: Complejidad de trabajo a realizar.
	\item Riesgo: Incertidumbre del trabajo a realizar.
\end{itemize}
Se utilizará una escala siguiendo la serie de Fibonacci, donde un número mayor implica mayor costo. Si la suma de los 3 componentes no da un número de la serie, se eligirá el próximo más cercano.

\begin{enumerate}
	\item Como operario quiero detectar la presencia de errores para reportarlo a mi supervisor.
	\begin{itemize}
		\item D: 8.
		\item C: 5.
		\item R: 3.
		\item Total: 21.
	\end{itemize}
	\item Como operario quiero una interfaz de usuario simple para evitar cometer errores.
	\begin{itemize}
		\item D: 3.
		\item C: 1.
		\item R: 1.
		\item Total: 5.
	\end{itemize}
	\item Como trabnajador de mantenimiento quiero conocer los problemas de los actuadores para repararlos rápidamente.
	\begin{itemize}
		\item D: 5.
		\item C: 5.
		\item R: 3.
		\item Total: 13.
	\end{itemize}
	\item Como desarrollador quiero controlar los actuadores para facilitar la programación.
	\begin{itemize}
		\item D: 3.
		\item C: 5.
		\item R: 3.
		\item Total: 13.
	\end{itemize}
	\item Como desarrollador quiero saber el estado de funcionamiento para facilitar la programación.
	\begin{itemize}
		\item D: 8.
		\item C: 5.
		\item R: 3.
		\item Total: 21.
	\end{itemize}
	\item Como programador de PLC quiero recibir señales del estado de actuadores para facilitar la programación.
	\begin{itemize}
		\item D: 3.
		\item C: 3.
		\item R: 1.
		\item Total: 8.
	\end{itemize}
	\item "Como gerente de planta, quiero minimizar la cantidad de sensores para disminuir los costos.
	\begin{itemize}
		\item D: 3.
		\item C: 3.
		\item R: 3.
		\item Total: 13.
	\end{itemize}
	\item Como gerente de planta, quiero que los componentes de las líneas de producción sean robustos para minimizar los tiempos de parada.
	\begin{itemize}
		\item D: 3.
		\item C: 5.
		\item R: 5.
		\item Total: 13.
	\end{itemize}
	\item Como gerente de planta, quiero que los actuadores de las líneas reporten errores para minimizar mantenimiento.
	\begin{itemize}
		\item D: 5.
		\item C: 5.
		\item R: 3.
		\item Total: 13.
	\end{itemize}
\end{enumerate}

\section{8. Entregables principales del proyecto}
\label{sec:entregables}

Los entregables del proyecto son:

\begin{itemize}
	\item Prototipo del sistema
	\item Manual de uso
	\item Diagrama de circuitos esquemáticos
	\item Diagrama de PCB
	\item Archivos de fabricación de PCB
	\item Diagrama de instalación
	\item Planos de caja
	\item Informe final
\end{itemize}

\section{9. Desglose del trabajo en tareas}
\label{sec:wbs}

A continuación se enumeran las tareas del proyecto y se detalla su carga horaria:

\begin{enumerate}
	\item \textbf{Planificación y gestión del proyecto (80 hs):}
	\begin{enumerate}
		\item Realizar el plan de trabajo (20 hs).
		\item Determinación de componentes (10 hs)
		\item Realizar informes de avance (10 hs)
		\item Confección de memoria de trabajo (30 hs)
		\item Presentación y defensa de trabajo (10 hs)
	\end{enumerate}
	\item \textbf{Tareas de investigación (60 hs):}
	\begin{enumerate}
		\item Busqueda de soluciones similares (20 hs)
		\item Estudiar protocolo CAN (20 hs)
		\item Examinar prestaciones del microcontrolador (10 hs)
		\item Recopilar hojas de datos de componentes y búsqueda de librerias (10 hs)
	\end{enumerate}
	\item \textbf{Desarrollo de software (185 hs):}
	\begin{enumerate}
		\item Integración de drivers y librerías (15 hs)
		\item Elaboración de estructura de mensajes (30 hs)
		\item Programación de red CAN (20 hs)
		\item Implementación de estructura de mensajes (20 hs)
		\item Programación de menúes (20 hs)
		\item Desarrollo de aplicación (40 hs)
		\item Integración de software de servomotores (40 hs)
	\end{enumerate}
	\textbf{\item Desarrollo de Hardware (95 hs):}
	\begin{enumerate}
		\item Diseño de circuitos eléctricos (30 hs)
		\item Diseño de PCB (30 hs)
		\item Diseño de red CAN (20 hs)
		\item Elaboración de archivos de fabricación (15 hs)
	\end{enumerate}
	\textbf{\item Fabricación (65 hs):}
	\begin{enumerate}
		\item Compra de componentes (20 hs)
		\item Fabricación de PCB (5 hs)
		\item Ensamble de PCB (20 hs)
		\item Diseño y fabricación de caja (20 hs)
	\end{enumerate}
	\textbf{\item Ensayos e implementación(145 hs):}
	\begin{enumerate}
		\item Pruebas eléctricas (15 hs)
		\item Ensayo de comunicación (40 hs)
		\item Ensayo de interfaz de usuario (20 hs)
		\item Prueba de aplicación (30 hs)
		\item Optimización y búsqueda de errores (20 hs)
		\item Pruebas de implementación en planta (20 hs)
	\end{enumerate}
\end{enumerate}

\textbf{Cantidad total de horas: (630 hs)}

\section{10. Diagrama de Activity On Node}
\label{sec:AoN}

\begin{consigna}{red}
Armar el AoN a partir del WBS definido en la etapa anterior. 

%La figura \ref{fig:AoN} fue elaborada con el paquete latex tikz y pueden consultar la siguiente referencia \textit{online}:

%\url{https://www.overleaf.com/learn/latex/LaTeX_Graphics_using_TikZ:_A_Tutorial_for_Beginners_(Part_3)\%E2\%80\%94Creating_Flowcharts}

\end{consigna}

\begin{figure}[htpb]
\centering 
\includegraphics[width=.8\textwidth]{./Figuras/AoN.png}
\caption{Diagrama en \textit{Activity on Node}}
\label{fig:AoN}
\end{figure}

Indicar claramente en qué unidades están expresados los tiempos.
De ser necesario indicar los caminos semicríticos y analizar sus tiempos mediante un cuadro.
Es recomendable usar colores y un cuadro indicativo describiendo qué representa cada color, como se muestra en el siguiente ejemplo:



\section{11. Diagrama de Gantt}
\label{sec:gantt}

\begin{consigna}{red}

Existen muchos programas y recursos \textit{online} para hacer diagramas de gantt, entre los cuales destacamos:

\begin{itemize}
\item Planner
\item GanttProject
\item Trello + \textit{plugins}. En el siguiente link hay un tutorial oficial: \\ \url{https://blog.trello.com/es/diagrama-de-gantt-de-un-proyecto}
\item Creately, herramienta online colaborativa. \\\url{https://creately.com/diagram/example/ieb3p3ml/LaTeX}
\item Se puede hacer en latex con el paquete \textit{pgfgantt}\\ \url{http://ctan.dcc.uchile.cl/graphics/pgf/contrib/pgfgantt/pgfgantt.pdf}
\end{itemize}

Pegar acá una captura de pantalla del diagrama de Gantt, cuidando que la letra sea suficientemente grande como para ser legible. 
Si el diagrama queda demasiado ancho, se puede pegar primero la ``tabla'' del Gantt y luego pegar la parte del diagrama de barras del diagrama de Gantt.

Configurar el software para que en la parte de la tabla muestre los códigos del EDT (WBS).\\
Configurar el software para que al lado de cada barra muestre el nombre de cada tarea.\\
Revisar que la fecha de finalización coincida con lo indicado en el Acta Constitutiva.

En la figura \ref{fig:gantt}, se muestra un ejemplo de diagrama de gantt realizado con el paquete de \textit{pgfgantt}. En la plantilla pueden ver el código que lo genera y usarlo de base para construir el propio.

\begin{figure}[htbp]
\begin{center}
\begin{ganttchart}{1}{12}
  \gantttitle{2020}{12} \\
  \gantttitlelist{1,...,12}{1} \\
  \ganttgroup{Group 1}{1}{7} \\
  \ganttbar{Task 1}{1}{2} \\
  \ganttlinkedbar{Task 2}{3}{7} \ganttnewline
  \ganttmilestone{Milestone o hito}{7} \ganttnewline
  \ganttbar{Final Task}{8}{12}
  \ganttlink{elem2}{elem3}
  \ganttlink{elem3}{elem4}
\end{ganttchart}
\end{center}
\caption{Diagrama de gantt de ejemplo}
\label{fig:gantt}
\end{figure}


\begin{landscape}
\begin{figure}[htpb]
\centering 
\includegraphics[height=.85\textheight]{./Figuras/Gantt-2.png}
\caption{Ejemplo de diagrama de Gantt rotado}
\label{fig:diagGantt}
\end{figure}

\end{landscape}

\end{consigna}


\section{12. Presupuesto detallado del proyecto}
\label{sec:presupuesto}

\begin{consigna}{red}
Si el proyecto es complejo entonces separarlo en partes:
\begin{itemize}
	\item Un total global, indicando el subtotal acumulado por cada una de las áreas.
	\item El desglose detallado del subtotal de cada una de las áreas.
\end{itemize}

IMPORTANTE: No olvidarse de considerar los COSTOS INDIRECTOS.

\end{consigna}

\begin{table}[htpb]
\centering
\begin{tabularx}{\linewidth}{@{}|X|c|r|r|@{}}
\hline
\rowcolor[HTML]{C0C0C0} 
\multicolumn{4}{|c|}{\cellcolor[HTML]{C0C0C0}COSTOS DIRECTOS} \\ \hline
\rowcolor[HTML]{C0C0C0} 
Descripción &
  \multicolumn{1}{c|}{\cellcolor[HTML]{C0C0C0}Cantidad} &
  \multicolumn{1}{c|}{\cellcolor[HTML]{C0C0C0}Valor unitario} &
  \multicolumn{1}{c|}{\cellcolor[HTML]{C0C0C0}Valor total} \\ \hline
 &
  \multicolumn{1}{c|}{} &
  \multicolumn{1}{c|}{} &
  \multicolumn{1}{c|}{} \\ \hline
 &
  \multicolumn{1}{c|}{} &
  \multicolumn{1}{c|}{} &
  \multicolumn{1}{c|}{} \\ \hline
\multicolumn{1}{|l|}{} &
   &
   &
   \\ \hline
\multicolumn{1}{|l|}{} &
   &
   &
   \\ \hline
\multicolumn{3}{|c|}{SUBTOTAL} &
  \multicolumn{1}{c|}{} \\ \hline
\rowcolor[HTML]{C0C0C0} 
\multicolumn{4}{|c|}{\cellcolor[HTML]{C0C0C0}COSTOS INDIRECTOS} \\ \hline
\rowcolor[HTML]{C0C0C0} 
Descripción &
  \multicolumn{1}{c|}{\cellcolor[HTML]{C0C0C0}Cantidad} &
  \multicolumn{1}{c|}{\cellcolor[HTML]{C0C0C0}Valor unitario} &
  \multicolumn{1}{c|}{\cellcolor[HTML]{C0C0C0}Valor total} \\ \hline
\multicolumn{1}{|l|}{} &
   &
   &
   \\ \hline
\multicolumn{1}{|l|}{} &
   &
   &
   \\ \hline
\multicolumn{1}{|l|}{} &
   &
   &
   \\ \hline
\multicolumn{3}{|c|}{SUBTOTAL} &
  \multicolumn{1}{c|}{} \\ \hline
\rowcolor[HTML]{C0C0C0}
\multicolumn{3}{|c|}{TOTAL} &
   \\ \hline
\end{tabularx}%
\end{table}


\section{13. Gestión de riesgos}
\label{sec:riesgos}

\begin{consigna}{red}
a) Identificación de los riesgos (al menos cinco) y estimación de sus consecuencias:
 
Riesgo 1: detallar el riesgo (riesgo es algo que si ocurre altera los planes previstos de forma negativa)
\begin{itemize}
	\item Severidad (S): mientras más severo, más alto es el número (usar números del 1 al 10).\\
	Justificar el motivo por el cual se asigna determinado número de severidad (S).
	\item Probabilidad de ocurrencia (O): mientras más probable, más alto es el número (usar del 1 al 10).\\
	Justificar el motivo por el cual se asigna determinado número de (O). 
\end{itemize}   

Riesgo 2:
\begin{itemize}
	\item Severidad (S): 
	\item Ocurrencia (O):
\end{itemize}

Riesgo 3:
\begin{itemize}
	\item Severidad (S): 
	\item Ocurrencia (O):
\end{itemize}


b) Tabla de gestión de riesgos:      (El RPN se calcula como RPN=SxO)

\begin{table}[htpb]
\centering
\begin{tabularx}{\linewidth}{@{}|X|c|c|c|c|c|c|@{}}
\hline
\rowcolor[HTML]{C0C0C0} 
Riesgo & S & O & RPN & S* & O* & RPN* \\ \hline
       &   &   &     &    &    &      \\ \hline
       &   &   &     &    &    &      \\ \hline
       &   &   &     &    &    &      \\ \hline
       &   &   &     &    &    &      \\ \hline
       &   &   &     &    &    &      \\ \hline
\end{tabularx}%
\end{table}

Criterio adoptado: 
Se tomarán medidas de mitigación en los riesgos cuyos números de RPN sean mayores a...

Nota: los valores marcados con (*) en la tabla corresponden luego de haber aplicado la mitigación.

c) Plan de mitigación de los riesgos que originalmente excedían el RPN máximo establecido:
 
Riesgo 1: plan de mitigación (si por el RPN fuera necesario elaborar un plan de mitigación).
  Nueva asignación de S y O, con su respectiva justificación:
  - Severidad (S): mientras más severo, más alto es el número (usar números del 1 al 10).
          Justificar el motivo por el cual se asigna determinado número de severidad (S).
  - Probabilidad de ocurrencia (O): mientras más probable, más alto es el número (usar del 1 al 10).
          Justificar el motivo por el cual se asigna determinado número de (O).

Riesgo 2: plan de mitigación (si por el RPN fuera necesario elaborar un plan de mitigación).
 
Riesgo 3: plan de mitigación (si por el RPN fuera necesario elaborar un plan de mitigación).

\end{consigna}


\section{14. Gestión de la calidad}
\label{sec:calidad}

\begin{consigna}{red}
Para cada uno de los requerimientos del proyecto indique:
\begin{itemize} 
\item Req \#1: copiar acá el requerimiento.

\begin{itemize}
	\item Verificación para confirmar si se cumplió con lo requerido antes de mostrar el sistema al cliente. Detallar 
	\item Validación con el cliente para confirmar que está de acuerdo en que se cumplió con lo requerido. Detallar  
\end{itemize}

\end{itemize}

Tener en cuenta que en este contexto se pueden mencionar simulaciones, cálculos, revisión de hojas de datos, consulta con expertos, mediciones, etc.  Las acciones de verificación suelen considerar al entregable como ``caja blanca'', es decir se conoce en profundidad su funcionamiento interno.  En cambio, las acciones de validación suelen considerar al entregable como ``caja negra'', es decir, que no se conocen los detalles de su funcionamiento interno.

\end{consigna}

\section{15. Procesos de cierre}    
\label{sec:cierre}

\begin{consigna}{red}
Establecer las pautas de trabajo para realizar una reunión final de evaluación del proyecto, tal que contemple las siguientes actividades:

\begin{itemize}
	\item Pautas de trabajo que se seguirán para analizar si se respetó el Plan de Proyecto original:
	 - Indicar quién se ocupará de hacer esto y cuál será el procedimiento a aplicar. 
	\item Identificación de las técnicas y procedimientos útiles e inútiles que se emplearon, y los problemas que surgieron y cómo se solucionaron:
	 - Indicar quién se ocupará de hacer esto y cuál será el procedimiento para dejar registro.
	\item Indicar quién organizará el acto de agradecimiento a todos los interesados, y en especial al equipo de trabajo y colaboradores:
	  - Indicar esto y quién financiará los gastos correspondientes.
\end{itemize}

\end{consigna}


\end{document}
